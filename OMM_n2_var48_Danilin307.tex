\documentclass{article}
\usepackage[utf8]{inputenc}
\usepackage{amsmath,amsthm,amssymb}
\usepackage{mathtext}
\usepackage[T1,T2A]{fontenc}
\usepackage[utf8]{inputenc}
\usepackage[english,bulgarian,russian]{babel}
\usepackage{graphicx}
\DeclareGraphicsExtensions{.pdf,.png,.jpg}

\title{Основы математического моделирования. Отсчет по практическому заданию №2}
\author{Рем Данилин группа 307}
\date{Вариант48}

\begin{document}

\maketitle
\section{Аналитическое решение}
\begin{equation}
    \begin{cases}
        \frac{\partial u }{\partial t} = \Delta u + \sin x \sin t \quad 0<x<\pi\quad 0<y<3\\
        u|_{x = 0} = u|_{x = \pi} = 0\\
        u|_{y = 0} = u|_{y = 3} = 0\\
        u|_{t = 0} = 0
    \end{cases}
\end{equation}
Из вида задачи (1) можно понять, что начальные условия и
неоднородность ортогональны. Поэтому будем искать решение в виде:
$$
u(x,y,t) = \sum\limits_{n=0}^{\infty}\sum\limits_{m=0}^{\infty} T_{nm} V_{nm}(x,y)
$$
Подставляя в уравнение получем следующую систему уравнений для поиска $T_{nm}$ и $V_{nm}$:
$$
\begin{cases}
\frac{d T_{nm}}{dt} + \lambda_{nm} T_{nm} = f_{nm}(t)\\
T_{nm} (0) = 0
\end{cases}
$$
$$
\begin{cases}
\Delta V_{nm} + \lambda_{nm} V_{nm} = 0\\
V|_{x=0} = V|_{x=\pi} = V|_{y=0} =V|_{y=3} = 0
\end{cases}
$$

Повторно разделяя переменные в последней системе для $V_{nm} = X_n Y_m$ приходим к следующему результату:
$$
\begin{cases}
X_{n} = \sin (nx) \\
Y_m = \sin (\frac{m\pi}{3})\\
\lambda_{nm} = n + \frac{m\pi}{3}
\end{cases}
$$

Преступим к решению задачи по поиску $T_{nm}$. Для этого, сначала необходимо получить явный вид функции $f_{nm}$. Его можно получить, вычислив интеграл
$$
f_{nm} = \frac{1}{|V_{nm}|^2} \int\int \sin x \sin t V_{nm} dx dy = \frac{4 \sin t}{3\pi} \delta_{1n}\frac{6}{\pi m}\quad \text{if m is odd}
$$
Получаем задачу Коши:
$$
\begin{cases}

\frac{d T_{nm}}{dt} +  \lambda_{nm}T_{nm} = 
\frac{4\sin t}{3\pi} \delta_{1n} \frac{6}{\pi m}\quad \text{if m is odd}\\
\lambda_{nm} = n + \frac{m\pi}{3}\\
T_{nm}(0) = 0
\end{cases}
$$

Решение задачи Коши для можно записать с использованием импульсной функции Коши:
$$
T_{nm} = \int\limits_{0}^{t} e^{-\lambda_{nm}(t-\tau)} \frac{4\sin \tau}{3\pi}\delta_{1n}\frac{6}{\pi m}d\tau =\frac{12}{\pi^2 m} \frac{ e^{-\lambda_{nm}t} - \cos t + \lambda_{nm}\sin t}{1 + \lambda_{nm}^2}\quad \text{if m is odd and n = 1}
$$
Остальные члены ряда, когда $n\neq 1$ и $m$- четное занулятся.


В итоге конечное решение будет представлять ряд:
$$
u(x,y,t) =  \sum\limits_{\text{m is odd}}^{\infty} \frac{12}{\pi^2 m} \frac{ e^{-\lambda_{nm}t} - \cos t + \lambda_{nm}\sin t}{1 + \lambda_{nm}^2}\quad \sin x \sin \left(\frac{\pi m y }{3}\right)
$$

Численно ряд был посчитан только до $1000$ члена, что в дальнейшем сильно повлияет на погрешность численного решения на концах области.





\section{Описание разностной схемы}
Так как в данной задаче мы имеем граничные условия Дерихле, то вводить в расчетной области равномерную сетку можно следующим способом.
$$
x_n = n h_x\quad n=0,1,2...N\quad h_x = \frac{1}{N}
$$
$$
y_m = m h_y\quad m=0,1,2...M\quad h_y = \frac{2}{M}
$$
$$
t_j = j \tau\quad j=0,1,2...J\quad \tau = \frac{T}{J}
$$
Начальные и граничные условия аппроксимируются точно. Граничные условия не зависят от времени. Пусть $\omega$ - сеточная функция, которая является решением.
\begin{equation}
    \begin{cases}
        \omega^{j}_{0m} = \omega^{j}_{Nm} = 0 \quad m=0,1,2...M\\
        \omega^{j}_{n0} = \omega^{j}_{nM} = 0 \quad 
        n=0,1,2...N\\
        \omega^0_{nm} = \sin 2\pi x_n \sin \pi y_m\quad n=0,1,2...N\quad m=0,1,2...M
    \end{cases}
\end{equation}
Предположим, что решение на слое $j$ нам известно найдем решение на слое $j+1$ в два этапа:



$$ j \rightarrow j+\frac{1}{2}$$




$$j + \frac{1}{2} \rightarrow j+1$$
Перейдем к рассмотрению \textbf{первого этапа}. С учетом того, что граничные условия \textbf{не зависят от времени}, систему для данного этапа решения можно написать следующим образом.
\begin{equation}
    \begin{cases}
    \frac{\omega_{nm}^{j+\frac{1}{2}} - \omega^j_{nm}}{0.5\tau} = \frac{1}{h_x^2}\left(
    \omega_{n+1,m}^{j+\frac{1}{2}} - 2 \omega_{nm}^{j+\frac{1}{2}} + \omega_{n-1,m}^{j+\frac{1}{2}}
    \right)
    +
    \frac{1}{h_y^2}\left(
    \omega^j_{nm+1} - 2\omega_{nm}^{j} + \omega_{nm-1}^{j}
    \right), m=1,2...M-1; n=1..N-1\\
    \\
    \omega_{0m}^{j+\frac{1}{2}} = 0 \quad m=1,2...M-1\\
    \omega_{Nm}^{j+\frac{1}{2}} = 0 \quad m=1,2...M-1
    \end{cases}
\end{equation}
Для каждого $m=1,2...M-1$ данную систему можно перезаписать следующим образом (для того, чтобы явно была видна структура трехдиогональной матрицы).
\begin{equation}
    \begin{cases}
    \omega_{0m}^{j+\frac{1}{2}} = 0\\
    \frac{1}{h_x^2} \omega_{n-1,m}^{j+\frac{1}{2}} -  (\frac{2}{\tau}+ \frac{2}{h_x^2}) \omega_{nm}^{j+\frac{1}{2}} + \frac{1}{h_x^2} \omega_{n+1,m}^{j+\frac{1}{2}}  = -\frac{1}{h_y^2}\left(
    \omega^j_{nm+1} - 2\omega_{nm}^{j} + \omega_{nm-1}^{j}
    \right) - \frac{\omega_{nm}^j}{0.5\tau}\\
    \omega_{Nm}^{j+\frac{1}{2}} = 0 
    \end{cases}
\end{equation}
\begin{equation}
    \begin{cases}
    \omega_{0m}^{j+\frac{1}{2}} = 0\\
    A^x \omega_{n-1,m}^{j+\frac{1}{2}} -  C^x \omega_{nm}^{j+\frac{1}{2}} + B^x \omega_{n+1,m}^{j+\frac{1}{2}}  = -F^x_n\\
    \omega_{Nm}^{j+\frac{1}{2}} = 0 
    \end{cases}
\end{equation}
Где $$A^x = \frac{1}{h_x^2}\quad B^x = \frac{1}{h_x^2}\quad C^x =  \frac{2}{\tau}+ \frac{2}{h_x^2} $$
$$
F^x_n = \frac{1}{h_y^2}\left(
    \omega^j_{nm+1} - 2\omega_{nm}^{j} + \omega_{nm-1}^{j}
    \right) + \frac{\omega_{nm}^j}{0.5\tau}
$$
Решаем данную систему численно методом прогонки.



n = 1
$$
-C^x\omega_{1m}^{j+\frac{1}{2}} + B^x \omega_{2m}^{j+\frac{1}{2}} = - F_1^x 
$$
Выражаем отсюда $\omega_{1m}^{j+\frac{1}{2}}$.
$$
\omega_{1m}^{j+\frac{1}{2}}  = \frac{B^x}{C^x} \omega_{2m}^{j+\frac{1}{2}} + \frac{F_1^x}{C^x} =  \alpha_1 \omega_{2m}^{j+\frac{1}{2}} + \beta_1
$$
n = 2
$$
A^x\omega_{1m}^{j+\frac{1}{2}} -C^x\omega_{2m}^{j+\frac{1}{2}} + B^x \omega_{3m}^{j+\frac{1}{2}} = - F_2^x 
$$
Подставляем ранее найденный $\omega_{1m}^{j+\frac{1}{2}}$ и выражаем $\omega_{2m}^{j+\frac{1}{2}}$.
$$
\omega_{2m}^{j+\frac{1}{2}} = \frac{B^x}{C^x - A^x\alpha_1} \omega_{3m}^{j+\frac{1}{2}} + \frac{F^x_2 + A^x\beta_1}{C^x - A^x\alpha_1} = \alpha_2 \omega_{3m}^{j+\frac{1}{2}} + \beta_2
$$
И так далее продолжаем подсчитывать коэффициенты $\alpha_n$ и $\beta_n$ по реккурентным формулам:
$$
\alpha_n = \frac{B^x}{C^x - A^x\alpha_{n-1}}\quad\beta_n = \frac{F_n^x + A^x \beta_{n-1}}{C^x - A^x\alpha_{n-1}}\quad \forall n = 1,2,...,N-1
$$
$$
\alpha_0 = \beta_0 = \alpha_N = \beta_N =0
$$
По коэффициентам восстанавливаем значения сеточной функции на слое $j + \frac{1}{2}$, используем обратную прогонку. 
\begin{equation}
\begin{cases}
\omega_{Nm}^{j+\frac{1}{2}} = \beta_N\\
\omega_{nm}^{j+\frac{1}{2}} = \alpha_n \omega_{n+1m}^{j+\frac{1}{2}} + \beta_n
\end{cases}
\end{equation}
Теперь переходим ко \textbf{второму этапу}, а именно переходу со слоя $j+\frac{1}{2} \rightarrow j+1$. Запишем разностную систему для этого перехода.
\begin{equation}
    \begin{cases}
    \frac{\omega_{nm}^{j+1} - \omega_{nm}^{j+\frac{1}{2}}}{0.5\tau} = \frac{1}{h_x^2}\left(\omega_{n+1 m }^{j+\frac{1}{2}} - 2\omega_{n m }^{j+\frac{1}{2}} + \omega_{n-1 m }^{j+\frac{1}{2}}   \right) + \frac{1}{h_y^2}\left(\omega_{n m+1 }^{j+1} - 2\omega_{n m }^{j+1} + \omega_{n m-1 }^{j+1}   \right)
    \\
    \omega_{n0}^{j+1} = 0
    \\
    \omega_{nM}^{j+1} = 0
    \end{cases}
\end{equation}
При каждом фиксированном $n = 1,2,...,N-1$ решаем систему, как и в прошлый раз, методом прогонки. Перепишем уравнение.
\begin{equation}
    \begin{cases}
    \omega_{n0}^{j+1} = 0
    \\
    \frac{1}{h_y^2} \omega_{n m+1 }^{j+1} - (\frac{2}{h_y^2} + \frac{2}{\tau})\omega_{n m }^{j+1} + \frac{1}{h_y^2}\omega_{n m-1 }^{j+1} = - \frac{\omega_{n m }^{j+\frac{1}{2}}}{0.5\tau} - \frac{1}{h_x^2}\left(\omega_{n+1 m }^{j+\frac{1}{2}} - 2\omega_{n m }^{j+\frac{1}{2}} + \omega_{n-1 m }^{j+\frac{1}{2}}   \right)
    \\
    \omega_{nM}^{j+1} = 0
    \end{cases}
\end{equation}

\begin{equation}
    \begin{cases}
    \omega_{n0}^{j+1} = 0\\
    A^y \omega_{n,m-1}^{j+1} -  C^y \omega_{nm}^{j+1} + B^y \omega_{n,m+1}^{j+1}  = -F^y_m\\
    \omega_{nM}^{j+1} = 0 
    \end{cases}
\end{equation}
Где 
$$
A^y = B^y = \frac{1}{h_y^2}
$$
$$
F^y_m = \frac{\omega_{n m }^{j+\frac{1}{2}}}{0.5\tau} + \frac{1}{h_x^2}\left(\omega_{n+1 m }^{j+\frac{1}{2}} -2\omega_{n m }^{j+\frac{1}{2}} + \omega_{n-1 m }^{j+\frac{1}{2}}   \right)
$$
Решаем данную систему численно методом прогонки.




m = 1
$$
-C^y\omega_{n1}^{j+1} + B^x \omega_{n2}^{j+1} = - F_1^y 
$$
Выражаем отсюда $\omega_{n1}^{j+1}$.
$$
\omega_{n1}^{j+1}  = \frac{B^y}{C^y} \omega_{2m}^{j+1} + \frac{F_1^y}{C^y} =  \alpha_1 \omega_{n2}^{j+1} + \beta_1
$$
m = 2
$$
A^y\omega_{n1}^{j+1} -C^y\omega_{n2}^{j+1} + B^x \omega_{n3}^{j+1} = - F_3^y
$$
Подставляем ранее найденный $\omega_{n2}^{j+1}$ и выражаем $\omega_{n2}^{j+1}$.
$$
\omega_{n2}^{j+1} = \frac{B^y}{C^y - A^y\alpha_1} \omega_{n3}^{j+1} + \frac{F^y_2 + A^y\beta_1}{C^y - A^y\alpha_1} = \alpha_2 \omega_{n3}^{j+1} + \beta_2
$$
И так далее продолжаем подсчитывать коэффициенты $\alpha_m$ и $\beta_m$ по реккурентным формулам:
$$
\alpha_m = \frac{B^y}{C^y - A^y\alpha_{m-1}}\quad\beta_m = \frac{F_m^y + A^y \beta_{m-1}}{C^y - A^y\alpha_{m-1}}\quad \forall m = 1,2,...,M-1
$$
$$
\alpha_0 = \beta_0 = \alpha_M = \beta_M =0
$$
По коэффициентам восстанавливаем значения сеточной функции на слое $j + 1$, используем обратную прогонку. 
\begin{equation}
\begin{cases}
\omega_{nM}^{j+1} = \beta_M\\
\omega_{nm}^{j+1} = \alpha_m \omega_{nm+1}^{j+1} + \beta_m
\end{cases}
\end{equation}
\textbf{Обоснование устойчивости}



Рассмотрим уравнения, полученные ранее (5) и (9):
$$
\frac{\omega_{nm}^{j+\frac{1}{2}} - \omega^j_{nm}}{0.5\tau} = \frac{1}{h_x^2}\left(
    \omega_{n+1,m}^{j+\frac{1}{2}} - 2 \omega_{nm}^{j+\frac{1}{2}} + \omega_{n-1,m}^{j+\frac{1}{2}}
    \right)
    +
    \frac{1}{h_y^2}\left(
    \omega^j_{nm+1} - 2\omega_{nm}^{j} + \omega_{nm-1}^{j}
    \right)
$$

$$
\frac{\omega_{nm}^{j+1} - \omega_{nm}^{j+\frac{1}{2}}}{0.5\tau} = \frac{1}{h_x^2}\left(\omega_{n+1 m }^{j+\frac{1}{2}} - 2\omega_{n m }^{j+\frac{1}{2}} + \omega_{n-1 m }^{j+\frac{1}{2}}   \right) + \frac{1}{h_y^2}\left(\omega_{n m+1 }^{j+1} - 2\omega_{n m }^{j+1} + \omega_{n m-1 }^{j+1}   \right)
$$
Выражаем $\omega^{1+\frac{1}{2}}$ из первого и подставляем во второе, также введем обозначение для разностных операторов.
$$
\Lambda_1 u = u_{\Bar{x}x} = \frac{u_{n-1m} - 2u_{nm} + u_{n+1m}}{h_x^2}
$$

$$
\Lambda_2 u = u_{\Bar{y}y} = \frac{u_{nm-1} - 2u_{nm} + u_{nm+1}}{h_y^2}
$$

В итоге получим
$$
\omega_t - \frac{\tau}{2}\Lambda_2\omega_t - \frac{\tau}{2}\Lambda_1\omega_t +\frac{\tau^2}{4}\Lambda_1\Lambda_2\omega_t = \Lambda_1 \omega^j + \Lambda_2 \omega^j
$$
Перепишем его немного в другом виде
$$
(E - \frac{1}{2}\tau\Lambda_1)(E - \frac{1}{2}\tau\Lambda_2)\omega_t = \Lambda \omega^j
$$
Исследуем схему на устойчивость по начальным условиям методом гармоник. Рассмотрим задачу Коши 
\begin{equation}
    \begin{cases}
    (E - \frac{1}{2}\tau\Lambda_1)(E - \frac{1}{2}\tau\Lambda_2)\omega_t = \Lambda \omega^j\\
    \omega^0_{nm} = e^{i(\alpha_qn + \beta_p m)}
    \end{cases}
\end{equation}
Тогда на слое $j+1$ решение будет иметь вид 
$$
\omega^j_{nm} = \lambda_{qp}^{j} e^{i(\alpha_qn + \beta_p m)}
$$
Найдем явный вид множителя роста, подставляя его в основное уравнение
$$
(E - \frac{1}{2}\tau\Lambda_1)(E - \frac{1}{2}\tau\Lambda_2)\frac{\lambda_{qp}^{j+1} - \lambda_{qp}^{j}}{\tau}e^{i(\alpha_qn + \beta_p m)} = \lambda_{qp}^{j}\Lambda e^{i(\alpha_qn + \beta_p m)}
$$
\begin{figure}[h]
\center{\includegraphics[width=1\linewidth]{plot/Screenshot_1.png}}
\end{figure}

\newpage
\begin{figure}[h]
\center{\includegraphics[width=0.7\linewidth]{plot/Screenshot_2.png}}
\end{figure}
То подставляя выражения и, сокращая на $\lambda_{pq}^je^{i(\alpha_qn + \beta_p m)}$, получаем
\begin{figure}[h]
\center{\includegraphics[width=1\linewidth]{plot/Screenshot_3.png}}
\end{figure}

\begin{figure}[h]
\center{\includegraphics[width=0.7\linewidth]{plot/Screenshot_4.png}}
\end{figure}

Отсюда следует, что множитель роста по модулю не превосходит единицу, поэтому данная схема является \textbf{безусловно устойчивой}.
\newpage
\section{Код программы}
\begin{verbatim}
from mpl_toolkits.mplot3d import Axes3D  # noqa: F401 unused import
import math
from scipy import optimize
import matplotlib.pyplot as plt
from matplotlib import cm
from matplotlib.ticker import LinearLocator, FormatStrFormatter
import numpy as np


max_x = math.pi
max_y = 3
TT = 0.07
N = 100
M = 100
J = 100
hx = max_x / N
hy = max_y / M
tau = TT/J

w = np.zeros((N+1,M+1,J+1))  # трехмерный массив для
# записи решения заполняется пока что нулями
# содержит только целые слои j и j+1

w_for_plot = np.zeros((N+1,M+1))

w05 = np.zeros((N+1,M+1))  # двумерный массив для
# записи решения заполняется пока что нулями
# содержит только значения на заданном полуцелом слое j+1/2

err = np.zeros((N+1,M+1)) # погрешность
err_for_plot = np.zeros((N+1,M+1))

analitic_sol = np.zeros((N+1,M+1))  # трехмерный массив для
# записи аналитического решения заполняется пока что нулями
# нужен для построения графика погрешности

analitic_sol_for_plot = np.zeros((N+1,M+1))# двумерный массив для
# построения аналитического решения в конечный момент времени

def Fy(w_plus1:float,w:float,w_minus1:float, x: float, t: float):
    return 1/hy**2 * (w_plus1 - 2*w + w_minus1) + 2*w/tau + math.sin(x) * math.sin(t)
def Fx(w_plus1:float,w:float,w_minus1:float, x: float, t: float):
    return 1/hx**2 * (w_plus1 - 2*w + w_minus1) + 2*w/tau + math.sin(x) * math.sin(t)
def b(i: int):
    return 12/((math.pi**2)*i)
def lmbd(i: int):
    return 1 + (math.pi*i)/3
def analitic(x: float,y: float,t: float):
    a = 0
    for i in range(1, 1000, 2):
        a += (math.sin(x)*math.sin(math.pi*y*i/3))*( b(i) * math.exp(-lmbd(i) * t) - b(i)*math.cos(t) + lmbd(i)*b(i)*math.sin(t) ) / (1+lmbd(i)**2)
    return a
def plot(X: list,Y: list,sol: np.ndarray):
    fig = plt.figure()
    ax = fig.gca(projection='3d')
    # Plot the surface.
    X, Y = np.meshgrid(X, Y)
    surf = ax.plot_surface(X, Y, sol, cmap='inferno', linewidth=0, antialiased=False)

    ax.set_xlabel('x')
    ax.set_ylabel('y')
    ax.set_zlabel('u')
    ax.set_zlim3d(0.0, 0.004)

    # Add a color bar which maps values to colors.
    fig.colorbar(surf, shrink=0.5, aspect=5)
    plt.show()
def plot_error(X: list,Y: list,sol: np.ndarray):
    fig = plt.figure()
    ax = fig.gca(projection='3d')
    # Plot the surface.
    X, Y = np.meshgrid(X, Y)
    surf = ax.plot_surface(X, Y, sol, cmap='inferno', linewidth=0, antialiased=False)

    ax.set_xlabel('x')
    ax.set_ylabel('y')
    ax.set_zlabel('u')
    ax.set_zlim3d(-0.1, 100.)
    # Add a color bar which maps values to colors.
    fig.colorbar(surf, shrink=0.5, aspect=5)
    plt.show()
def start_condition(x: float, y: float):
    return  0


# Создание сетки
X = [i*hx for i in range(N+1)]
Y = [i*hy for i in range(M+1)]
T = [i*tau for i in range(J+1)]
# -------------------------------------------------------
# Аналитическое решение построение
max_u = 0
for i in range(N+1):
    for j in range(M+1):
        a = analitic(X[i], Y[j], T[J])
        analitic_sol[i][j] = a
        if max_u < a:
            max_u = a
#max_u = 0
#for i in range(N+1):
#    for j in range(M+1):
#        analitic_sol_for_plot[i][j] = analitic(X[i], Y[j], T[J])
#        if max_u < analitic(X[i], Y[j], T[J]):
#            max_u = analitic(X[i], Y[j], T[J])

# -------------------------------------------------------
# Численное решение
# зададим начальные условия
for i in range(N+1):
    for j in range(M+1):
        w[i][j][0] = start_condition(X[i], Y[j])
# коэффициенты в ситсемах
Ax = 1/hx**2
Ay = 1/hy**2
Bx = 1/hx**2
By = 1/hy**2
Cx = 2/tau + 2/hx**2
Cy = 2/tau + 2/hy**2
# прогоначные коэффцициенты
alpha_x = [0 for i in range(N)]
beta_x = [0 for i in range(N)]
alpha_y = [0 for i in range(M)]
beta_y = [0 for i in range(M)]

for j in range(0,J):
    # переход на слой j+1/2
    for m in range(1,M):
        alpha_x[0] = 0
        beta_x[0] = 0
        # прямой ход прогонки
        for n in range(1,N):
            F = Fy(w[n][m+1][j], w[n][m][j], w[n][m-1][j], X[n], T[j] + tau/2 )
            beta_x[n] = (F + Ax * beta_x[n-1]) / (Cx - Ax * alpha_x[n-1])
            alpha_x[n] = Bx / (Cx - Ax*alpha_x[n-1])
        # обратный ход прогонки
        w05[N][m] = 0
        for n in range(N-1,-1,-1):
            w05[n][m] = alpha_x[n]*w05[n+1][m]+beta_x[n]

    # переход на слой j+1
    for n in range(1,N):
        alpha_y[0] = 0
        beta_y[0] = 0
        # прямой ход прогонки
        for m in range(1, M):
            F = Fx(w05[n+1][m], w05[n][m],w05[n-1][m], X[n], T[j] + tau/2 )
            beta_y[m] = (F + Ay * beta_y[m- 1]) / (Cy - Ay * alpha_y[m - 1])
            alpha_y[m] = By / (Cy - Ay * alpha_y[m - 1])
        # обратный ход прогонки
        w[n][M][j+1] = 0
        for m in range(M-1,-1,-1):
            w[n][m][j+1] = alpha_y[m] * w[n][m+1][j+1] + beta_y[m]
            if j == J-1:
                err[n][m] = abs( analitic_sol[n][m] - w[n][m][j+1] )*100/max_u


for i in range(N+1):
    for j in range(M+1):
        w_for_plot[i][j] = w[i][j][J]


plot(X, Y, analitic_sol)
plot(X, Y, w_for_plot)
plot_error(X, Y, err)
\end{verbatim}
\newpage
\section{Результат работы программы}
\begin{figure}[h]
\center{\includegraphics[width=1\linewidth]{plotTask2/danalitic01.png}}
\caption{аналитическое решение t = 0.1}
\label{ris:image}
\end{figure}
\begin{figure}[h]
\center{\includegraphics[width=1\linewidth]{plotTask2/dnum01.png}}
\caption{численное решение t = 0.1}
\label{ris:image}
\end{figure}

\begin{figure}[h]
\center{\includegraphics[width=1\linewidth]{plotTask2/derror01.png}}
\caption{ошибка  t = 0.1}
\label{ris:image}
\end{figure}

\begin{figure}[h]
\center{\includegraphics[width=1\linewidth]{plotTask2/danalitic007.png}}
\caption{аналитическое решение t = 0.007}
\label{ris:image}
\end{figure}
\begin{figure}[h]
\center{\includegraphics[width=1\linewidth]{plotTask2/dnum007.png}}
\caption{численное решение t = 0.007}
\label{ris:image}
\end{figure}

\begin{figure}[h]
\center{\includegraphics[width=1\linewidth]{plotTask2/derror007.png}}
\caption{ошибка  t = 0.007}
\label{ris:image}
\end{figure}




\end{document}